\documentclass{article}
\usepackage[utf8]{inputenc}
\usepackage[portuguese]{babel}
\usepackage{csquotes}
\usepackage{graphicx}
\usepackage{adjustbox}
\usepackage{lipsum}
\usepackage[backend=biber,autolang=other,
  bibencoding=utf8,style=alphabetic,
  ibidtracker=true]{biblatex}
\addbibresource{holanda.bib}

\title{Francisco de Holanda de Saint-Denis}
\date{20 de Abril de 2015}
\author{João Távora \\Faculdade de Belas Artes da Universidade de Lisboa}

\begin{document}

\maketitle

\section{Sinopse}

\section{Palavras-chave}

Holanda, Idade Média

\section{Introdução}

Francisco da Holanda escreve o livro ``A ciência do Desenho'' em 9999
e dedica-o, no próprio extensíssimo título da obra, a El-Rei
D. Sebastião como ``lembrança ao sereníssimo e cristianíssimo Rei
[...] de quanto serve a ciência do desenho [...]  assim na paz como na
guerra''. 

Esta obra ``A ciência do Desenho'', como é frequentemente abreviada,
estrutura-se de forma simples em 7 pequenos capítulos. Introduzida no
prólogo a motivação da escrita, o primeiro capítulo pretende ser um
levantamento, em jeito de alerta, dos avanços conseguidos pelos outros
reinos nesta área.

Estabelecida a urgência e pertinência da lembrança, o segundo capítulo
apresenta uma concepção inovadora e enaltecedora do ``DESENHO''
enquanto actividade intelectual, contrastando-a com o ``debuxo''
associado à simples prática manual. Os capítulos seguintes enumeram
sistematicamente a utilidade aplicada do desta ciência nos diversos
contextos práticos que o autor considera que possam interessar
directamente a El-Rei.

Assim, o desenho é útil, por esta ordem, no serviço Deus, à pessoal
real, na paz e na guerra. 

Na dimensão, a obra é assumidamente pequena, como é directamente
assumido no título do último capítulo ``Conclusão desta pequena
obra''.

Francisco da Holanda ilustra amplamente as suas ideias e posições com
episódios da sua própria vivência, bem como variada evidência
anedótica de difícil verificação \footnote{onde?}. As referências
auto-bibliográficas são tão numerosas que, na edição consultada
(\cite{holanda}), José da Felicidade Alves, se inclui até uma seçcão
exclusivamente dedicada a enumerá-las, contando-se em certas páginas
do manuscrito original mais de uma dezena delas.

\section{O queixume e os ``outros reinos''}

As primeiras duas palavras da obra do Prólogo d' ``A Ciência do
Desenho'' são ``Um queixume'':

\begin{quote}
  Um queixume faz por mim a Arte da Pintura a Vossa Alteza [...], de
  quão pouco ]e bem entendida e estimada, neste vosso Reino de
    Portugal, sendo uma ciência e arte digníssima [...] E somente em
    portugal não é conhecida nem tem o resplendor e lustro que merece.
\end{quote}

Mais adiante, Holanda esclarecerá que não é por ``ressábio'' que se
queixa \cite[fl.36v]{holanda}, mas pela benévola motivação de alertar
o mais alto responsável do Reino de como periga o futuro do reino se
não se compreenderem e promoverem devidamente as artes.

O leitor d'``A ciência do Desenho'' depara-se por toda a obra com uma
marcação deliberada da autoridade do autor, sobretudo referências aos
trabalhos realizados por Holanda aos antepassados de D. Sebastião, bem
como referências ao seu próprio pai, António Dolanda que seria
protegido do imperador Carlos V.

Assim, no primeiro capítulo Holanda começa por estabelecer a amplitude
do seu conhecimento na matéria \cite[fl.34r]{holanda}:

\begin{quote}
  [...] porque as li, e vi e sei e tratei [...] que me atrevera a
  encher muitos livros [...]
\end{quote}

Afirma novamente a brevidade da presente obra, e que tratará nela
apenas de uma pequena fracção da centena de pintores e artistas que
conheceu e admira.

\begin{quote}
  [...] haver piedade dela e dos que não entendem o preço de tão
  ilustríssima ciência, determinei de escrever este breve caderno
  acerca do valor que tem a Arte do Desenho da Pintura na República
  Cristã assim no tempo da paz como no tempo da guerra.
\end{quote}

Segundo Holanda, a pintura (e a do desenho, como virá a pretender) tem
``origem divina'', e por essa razão, sempre foi muito estimada por
``todas as repúblicas famosas e regidas com policia não bárbara''.

A pintura e o desenho, neste ponto do livro ainda indistintos, são
enquadrados numa moldura divina e com certa complexidade
mística \footnote{É plausível que a formulação algo retorcida desta
  frase tenha provindo da ``censura benévola'' de Frei Bartolomeu
  Ferreira, ainda que esta folha não apresente as emendas encontradas
  noutras folhas por José da Felicidade Alves.}. Afirma-se que estas
artes ``deriva[m] de [S]ua eterna origem a ideia dalgum grande engenho
no entendimento'' \cite[fl.34r]{holanda}. Uma vez estabelecida no
plano transcendental, a ideia da eternidade é repetido no plano
político já que o desenho ``em todas as idades e nações do mundo
sempre foi e é hoje muito estimado'' \cite[fl.34r]{holanda}

Entre ``outros reinos'' eleitos como exemplos, citam-se os de
Alexandre, de Antíoco, e o de César. Citam-se os pintores Protógenes e
Apeles \footnote{Crê-se que o elogio ao pintor Apeles provinha
  totalmente da celebridade do mesmo, já que deste não se conhece
  nenhuma obra em concreto. \cite{calado}} e Panfilo, preocupando-se
em enquadrar as relações privilegiadas que estes artistas mantinham
com os seus soberanos. Neste ponto, Holanda faz questão de refutar que
apenas os antigos, ``gentios e pagãos'' se preocupariam com as artes,
estabelecendo a ligação com a época cristã. Refere Leonardo da Vinci ,
que terá morrido nos braços do Rei de França, bem como as as ligações
de Rafael e Miguel Ângelo com o papado.

Esta folha d'``A Ciência do Desenho'' ilustra a lado lúdico e
anedódito da obra: quanto a Rafael, refere-se uma
historieta \footnote{Colhida nas ``Vidas'' de Vasari, segundo José da
  Felicidade Alves, em nota da edição, e portanto de fiabilidade
  relativa. } segundo a qual Rafael se teria mantido solteiro na
esperança que o papa o fizesse cardeal. Já de Miguel Ângelo, que
Holanda estimava particularmente e com quem teria mantido contacto
pessoal, conta-se a anedota ainda mais bizarra de que o pintor,
célebremente irascível, ``tirou quase uma tábua que houvera de
escalavrar o Papa'' quando se sentiu desrespeitado no seu estúdio.

\section{O desenho como ideia}

É notória em toda a obra a insistência e prevalência do termo
``ciência'' a preceder ``Desenho'' e ``Pintura'', ora só, ora
emparelhado com ``arte'', ou seja, Holanda esforça-se desde o início
para fazer descolar estes termos das ideias de manualidade e
artesanato. Ainda assim, no segundo capítulo da obra de Holanda que
reside e se desenvolve o núcleo teórico da obra, a definição inovadora
do ``Desenho'' como ideia e actividade intelectual.







\section{Serviço a Deus, paz e guerra }

``Na túmulo que é a última obra''.

\section{Conclusão}

O tom geral da obra, que Holanda nunca escamoteia, é marcadamente o
ressentimento de homem desencantado. Holanda refere frequentemente que
o seu espírito de artista se encontra ``de todo arrefecido [...] e
perdido, pelo tempo e lugar em que [hoje vive] no Monte, tratando
doutra Pintura''.

Pode especular-se que a evidente motivação deste tom de intimidade e
das inúmeras referências bilbiográficas a Holanda e aos feitos do seu
pai é a obtenção de alguma espécie de reconhecimento Real. Por outro
lado, pode tratar-se apenas de uma convenção estilística vigente na
época.

De todas as formas, Holanda é bastante duro e mordaz nas críticas. Das
``grandes nações'' que estimam o Desenho, somente Portugal ``que não
sabe agora mais disso que de coisa que nunca veio a sua notícia'',
``estima muito qualquer outra coisa que pintura nem pintores''. Estas
afirmações trocistas são certamente tecidas por alguém com uma certa
confiança na \emph{entourage} real.


Tom didático, paternalista, lobby da guerra, D. Sebastião visto como
um puto xarila, autovalorização

\printbibliography[heading=bibliography,title={Bibliografia}]

\end{document}
