
\documentclass{article}
\usepackage[utf8]{inputenc}
\usepackage[portuguese]{babel}
\usepackage{csquotes}
\usepackage{graphicx}
\usepackage{adjustbox}
\usepackage{lipsum}
\usepackage[backend=biber,autolang=other,
  bibencoding=utf8,style=alphabetic,
  ibidtracker=true]{biblatex}
\addbibresource{suger.bib}

\title{O abade de Suger}
\date{18 de Janeiro de 2014}
\author{João Távora \\Faculdade de Belas Artes da Universidade de Lisboa}

\begin{document}

\maketitle

\section{Sinopse}

\section{Palavras-chave}

Desenho, Villard de Honnecourt, Idade Média

\section{Introdução}

\section{Aspectos biográficos}

As origens da família de Suger são desconhecidas. Por diversas vezes
nos seus escritos, Suger sugere a proviniência de um meio humulde,
embora isso possa ser uma convenção da escrita autobiográfica. Em
1091, aos 10 anos de idade, Suger foi cedido como oblato à abadia de
St. Denis, onde principiou a sua educação tendo sido mais tarde
aprendiz no priorado de Saint-Denis de lÉstrée. Terá sido aqui que
conheceu aquele que viria a ser o rei de França Luís VI, também
conhecido como Luís, o Gordo. Depois de ter frequentado uma escola
vizinha de 1104 até 1106, Suger tornou-se secretário do abade de
St. Denis. Nos ano seguinte tornou-se preboste de Berneval na
Normandia, e em 1009 de Toury. Em 1118, o rei de França enviou Suger
como conselheiro à corte do Papa Gelásio II, em Maguelonne, que tinha
acabo de regressar a França depois de turbulências em Roma. Suger
viveu até 1122 na corte do sucessor de Gelásio, Calisto II.

No seu regresso, Suger tornouse abade de St. Denis. Até 1127,
ocupou-se principalmente com os assuntos do reino, mas durante a
década seguinte, dedicou-se à reorganização e reforma de St. Denis. Na
seçcão \REF{FIXME} trataremos em detalhe dos trabalhos encomendados
por Suger.

Em 1137, acompanhou aquele que viria a ser o rei Luís VII, também
chamado Luís, o Jovem, até à Aquitânia, na ocasião do casamento desse
príncipe com Leonor de Aquitânia. Na ausência do rei durante a Segunda
Cruzada, de 1147 até 1149, foi regente do reino de França. Este
casamento traria para o Reino de França o vasto condado da Aquitânia,
e Suger opôs-se à dissolução do casamento em 1152, o que acabou por
ocorrer pelo mesmo não ter produzido herdeiro.

\section{Política: Influência e Regência}

Luís VII, sucessor de Luís VI, passou grande parte da sua juventude em
Saint-Denis, onde aprendeu a confiar e a valorizar as opiniões do
abade Suger, que seria um bom conselheiro durante os primeiros anos do
seu reinado.

Suger roga a Luís VII que destrua os movimentos ameaçadores dos
senhores feudais, é responsável pela estratégia real 

\subsection{Resistência ao Imperador}

\subsection{Relação com S. Bernardo de Claraval}

\section{Reconstrução de St. Denis}

\subsection{Influências estéticas}

\section{Conclusão}


\section{Escritos}

\section{Conclusão}

\printbibliography[heading=bibliography,title={Bibliografia},type=book]

\end{document}
