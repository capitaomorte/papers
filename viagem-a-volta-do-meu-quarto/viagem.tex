\documentclass[12pt]{article}
\usepackage{fontspec}
\usepackage{titling}
\usepackage[portuguese]{babel}
\renewcommand{\baselinestretch}{1.5} 
\usepackage{csquotes}
\usepackage{graphicx}
\usepackage{adjustbox}
\usepackage{lipsum}
\usepackage[backend=biber,autolang=other,
  bibencoding=utf8,style=authoryear-ibid,
  ibidtracker=true]{biblatex}

\setmainfont[]{Times New Roman}

\usepackage[margin=2cm,left=3cm]{geometry}

\addbibresource{viagem.bib}

\title{Viagem à volta do meu quarto} \date{10 de
  Dezembro de 2015} \author{João Távora \\Faculdade de Belas Artes da
  Universidade de Lisboa}

\begin{document}

\maketitle

\section{Resumo}

Analisa-se brevemente ``Viagem à volta do meu quarto'', livro de 1795 de Xavier de Maistre, procurando nele pistas sobre a processos  envolvidos na prática artística, desde então até ao presente, explorando ligações entre esse livro e autores da filosofia, da poesia e das artes plásticas.

\section{Introdução}

``Viagem à volta do meu quarto'' é o título que Xavier de Maistre deu ao livro que escreveu em 1795 e no qual descreve o período de 42 dias de prisão domiciliária que lhe é imposto por envolvimento num duelo. O livro é um diário desse cárcere, escrito em jeito de romance de viagems. Trata-se de uma paródia da literatura deste género, particularmente estimada na época em que o ``Grand Tour'' europeu era um ritual iniciático fundamental na vida de qualquer jovem aristocrático. A este livro, de Maistre somou em 1825 outro, chamado ``Expedição noctura à volta do meu quarto''. Ambos se encontram estruturados em tantos capítulos quantos dias duraram as respectivas viagens.

Param além de comédias, estes livros são uma defesa da imaginação e da intimidade intelectual como fontes de descoberta. Escritos durante a turbulência da revolução francesa, na eclosão da modernidade, influenciaram escritores e artistas plásticos até ao presente. Este texto tem como objecto os aspectos da obra de Maistre que possam constituir incursões nas artes plásticas, sobretudo na Pintura e no Desenho.

\section{Desenvolvimento}

\subsection{Da paródia à inquietação filosófica}

Nas primeiras linhas de Maistre anuncia imodestamente que ``aparece de repente no mundo erudito'' com um ``livro de descobertas na mão''. Viajar no próprio quarto é ``um meio seguro contra o tédio'' e a cómoda e democrática forma de viajar: serve os indolentes, os indigentes, os medrosos.

O principal dispositivo lúdico é a insólita precisão do simulacro de viagem: de forma perfeitamente convincente descrevem-se levantamentos topológicos, reuniões, separações e imprevistos. Habilmente, expõe-se na descrição de cada um acontecimentos o ridículo do viajante fechado no seu quarto, e, de alguma forma, algum desse ridículo transfere-se para o referente real.

No entanto, ao contrário da paródia habitual, e pela disparidade inusitada da comparação, o segundo transforma-se facilmente em inquietação filosófica. Nesta dimensão quase sinistra - não faz diferença nenhuma se viajamos pelo nosso quarto ou por qualquer país exótico - a viagem de de Maistre transforma-se em muito mais que um \emph{pastiche} da literatura de viagens.

O livro é tanto mais rico quanto a exploração da ideia de viagem é apenas uma das suas possiblidades de leitura. Quanto a ela, encontra-se na obra livro tanto a sua negação categórica (Queirós, 2015) quanto a afirmação da sua essência. Mas, encontram-se outros estratos temáticos: no prefácio à edição portuguesa, Pedro Mexia ressalta, por exemplo, o a ideia de ``exílio íntimo'', e a dimensão política do livro escrito à sombra da ``tirania'' da Revolução Francesa.

Convém referir que se há inquietações filosóficas nestas leituras, elas são de natureza mais latente. Porque também as há de natureza mais explícita, como é caso da repetida oposição entre a alma do protagonista e a sua existência material(o ``animal''), numa espécie de falseto neoplatónico (de novo a paródia) que o autor detalha no capítulo VI, aquele que é destinado ``tão-só e apenas aos metafísicos''.

\subsection{Das torradas à Pintura}

Ainda que de Maistre investa bastante da sua viagem em considerações sobre as artes, não se pode dizer que a Pintura ocupe nelas um lugar cimeiro, pelo menos não de forma manifesta.

Ainda assim, quando não se está a inquietar filosoficamente ou a queimar o seu animal na tenaz da fogueira, podemos encontrar de Maistre evocando ideias visuais: descrevem-se os padrões abstractos de luz e sombra que as árvores lá fora produzem na parede, a cor perfeita da cama, contemplam-se retratos e estampas, dialoga-se com Rafael (ONDE?), invoca-se Apeles (ONDE?). Ao examinar a correspondência antiga na sua escrivaninha, apreciam-se as ``linhas traçadas'', na caligrafia executada pela ``mão conduzida pelo coração''.

Na levantamento topológico rigoroso do quarto, na intensa da cama, na alegria ao ver uma ``pirâmide de torradas'' há uma espécie de reverência misteriosa ao objecto mundano.

A propósito dela, não escaparam a Guiliana Bruno, professora de estudos visuais em Harvard, ligações entre a ``Viagem'' e as artes plásticas. Descreve um museu em França, o museu Marés, que o escultor Frederic Marés (1893-1991) fundou e baptizou de ``Museu Sentimental''. A autora fala do ``espectáculo das coisas que não transportam outro valor senão o seu poder emocional - objectos transformados em narrativas'' e de como esse museu incluenciou a arte contemporânea. Nesta medida o quarto de de Maistre situa-se no coração deste museu. O seu livro é uma ``viagem sentimental'' na ``eclosão da modernidade'' (Bruno 2002, p.133). Mais à frente, descreve o trabalho ``Atlas'' de Gehrard Richter como um ``arquivo de espaço íntimo'' que se radica em grande parte na obra de Maistre. (Bruno 2002)

Durante o século XX, de Egon Schiele a pintores menos conhecidos como John Bratby, pudemos contemplar esse espaços em pinturas dos quartos que os artistas habitaram. No entanto, talvez as pinturas que Vincent Van Gogh fez do seu modesto quarto em Arles \ref{fig:1} sejam as mais precoces e mais conhecidas. \footnote{A imagem de um destes quadros faz precisamente a capa de uma edição anterior da ``Viagem'' (QUAL?)}.

\begin{figure}
  \label{fig:1}
  \caption{Quarto de Van Gogh em Arles}
\end{figure1}

Numa carta ao seu irmão Theo tenta descrever a composição: ``Enfim la vue du tableu doit reposer la tête''. Ao reler a carta acrescenta de um modo igualmente enigmático e em letra mais pequena ``ou plutôt la imagination'' (Van Gogh 1988). Não sabemos se Van Gogh conheceu de Maistre, mas talvez possamos dizer que, fechado nesse quarto por doença \cite{wikipedia}, se conceberam estes quadros em condições muito semelhantes às do escritor. 

\subsection{Do outeiro ao acidente}

Em última instância, o tema da ``Viagem à volta do meu quarto'' é a construção do próprio livro. Logo após o início triunfalista, morre a vã esperança de alguma estabilidade estrutural que pudesse provir da coincidência entre os 42 dias de cárcere e os 42 capítulos, como se de uma viagem normal se pudesse vir a tratar, em que se percorresse o espaço físico do quarto com lentidão ou minúcias absurdas. Pelo contrário, tudo é inconstante, todas as agitações metafísicas ou sentimentais do protagonista ecoam-se na forma do livro: a única coisa que subsiste na coincidência capítulo-dia é a disparidade entre todos eles.

Como dias, há capítulos puramente indolentes e outros enérgicos e resolutos. Há capítulos muito curtos onde nada mais se faz do que esperar pelo próximo ou se evoca o capítulos passados. 

Cada capítulo do livro é um esboço inacabado que conta a história da sua própria concepção, como um desenho. E cada capítulo é um atentado contra todos os outros.

Por exemplo, o capítulo XII da ``Viagem...'' é sem dúvida o mais insólito. Ei-lo, integralmente:

\begin{quote}
  . . . . . . . . . . . . . . .\\
  . . . . . . . . . . . . . . .\\
  . . . . . . o outeiro . . . . . .\\
  . . . . . . . . . . . . . . .\\
  . . . . . . . . . . . . . . .\\
  . . . . . . . . . . . . . . .\\
\end{quote}

``O outeiro'' de que fala de Maistre é o local que recordou, no
capítulo anterior em que limpava o pó a um retrato, de ver pela última vez um amor predilecto. 

A emoção é tão forte, que esse outeiro acaba por eclipsar todos os seus outros ``pensamentos desconexos'' e mesmo o capítulo seguinte XIII (``os meus esforços são baldados [...] é uma etapa militar''). (de Maistre, p???)

De Maistre transforma o outeiro, pequena elevação de terreno, possivelmente o mais modesto dos acidentes geográficos, numa poderosa fonte de ausência e de assombro, que não só convida o leitor a imagina-lo como quase o obriga. Enquanto outros seus contemporâneos românticos escreverão sobre montanhas nevadas ou planícies verdejantes, de Maistre abriga tudo num outeiro, ou seja em nada. E isto é um paradoxo perfeitamente consistente com o da viagem imóvel, mas talvez ainda mais poderoso.

O poder expressivo do ``nada'' é curiosa predileção de algum poetas de língua portuguesa do século XX \cite{manoel}:

\begin{quote}
  ``Há tanta suavidade em nada se dizer / E tudo se entender — ''
\end{quote}
Fernando pessoa

\begin{quote}
``Há muitas maneiras sérias de não dizer nada, mas só a poesia é verdadeira''
\end{quote}


Este poeta brasileiro tem também um poema a propósito de outro acidente geográfico, que se pode ouvir \emph{on-line} dito pelo próprio autor.

\begin{quote}
  O rio que fazia uma volta atrás de nossa casa
  Era imagem de um vidro mole que fazia uma volta atrás de casa
  Passou um homem depois e disse:
  -Essa volta que o rio faz por trás de sua casa se chama enseada
  Não era mais a imagem de uma cobra de vidro que fazia uma volta atrás de casa. Era uma enseada.

  Acho que o nome empobreceu a imagem.
\end{quote}}

  Ainda que aqui seja, inversamente a de Maistre, o nome que devora a imagem, este poema encontra-se exactamente no mesmo lugar poético que o ``outeiro'': afirma a elipse como figura de estilo e a ausência como fonte imagética.

  A própria ideia de \emph{acidente geográfico} não é inocente. Um \emph{acidente} não é um acontecimento qualquer. No limite não pode ser descrito - da mesma forma que, ainda que de Maistre o tenha tentado quando examinava velha correspondência, não se pode descrever um traço ou uma pincelada. 

Martin Heidegger, que por coincidência também escrevia em retiro, concebe uma ideia semelhante n'``A Origem da Obra de Arte'', e precisamente acerca de uns sapatos de camponês pintados pelo pintor de quartos que apontámos no capítulo anterior (Heidegger, 1950):

\begin{quote}
  A partir da pintura de Van Gogh não podemos sequer estabelecer onde se encontram estes sapatos.[...] Um par se sapatos de camponês e nada mais. E todavia...
  Na escura abertura do interior gasto dos sapatos, fita-nos a dificuldade e o cansaço dos passos do trabalhador.
\end{quote}

Ou seja o nada do outeiro, como abertura escura, não só existe como nos fita e nos interpela.

Por fim, é o próprio de Maistre que, na ``Expedição nocturna...'' sente a necessidade consagrar um capítulo à sua estratégia compositiva acidental.

\begin{quote}
  Tinha uma velha parente, mulher de muita inteligência, cuja conversa era das mais interessantes; a sua memória, porém, inconstante e fértil a um tempo, fazia-a passar muitas vezes de um episódio a outro e de uma divigação a outra, a ponto de ver-se obrigada a implorar a ajuda de quem a ouvia: ``Que é que vos queria contar?'', dizia ela, e muitas vezes também os seus ouvintes se tinham esquecido, o que deixava toda a gente num apuro difícil de expressar. Ora, pôde notar-se que o mesmo acidente se verifica frequentemente nas minhas narrações, e devo convir, efectivamente, que o plano e a ordem da minha viagem são exactamente decalcados da ordem e do plano das conversas da minha tia; mas não peço o auxílio de ninguém, pois reparei que o assunto volta por si mesmo e no momento que menos espero.
\end{quote}

\section{Conclusão}

Com este texto espero ter demonstrado que à viagem imóvel, à paródia lúdica, à política e ao exílio sentimental, que a todos estes estratos que se econtram na ``Viagem à volta do meu quarto'' se pode somar outro de natureza puramente estética: o livro é uma espécie de manual de composição, um manual de espeleologia da própria imaginação, do qual se retiram consequências para a criação plástica em qualquer das suas linguagem, mas talvez mais para o desenho e a pintura, tanto mais que o a indepedência do tema é afirmada repetidamente pelo paradoxos e acientes.

Mas, se nas viagens que fizemos por ele o quarto de Maistre nos fitou com este aspecto de manual de pintura ou caderno de desenhos, outras viagens serão possíveis. Aliás, como se observa no prefácio, as diferenças entre a ``Viagem...'' e a ``Expedição nocturna...'', mostra precisamente que uma viagem, como um acidente, é uma coisa irrepetível. 


\printbibliography[heading=bibliography,title={Bibliografia}]

\end{document}

%% Local Variables:
%% coding: utf-8
%% End:
