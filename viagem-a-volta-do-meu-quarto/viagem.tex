\documentclass[12pt,a4paper]{article}
\usepackage{fontspec}
\usepackage{titling}
\usepackage[portuguese]{babel}
\renewcommand{\baselinestretch}{1.5} 
\usepackage{csquotes}
\usepackage{graphicx}
\usepackage{adjustbox}
\usepackage{lipsum}
\usepackage[backend=biber,autolang=other,
  bibencoding=utf8,style=authoryear-ibid,
  ibidtracker=true]{biblatex}

\setmainfont[]{Times New Roman}

\usepackage[margin=2cm,left=3cm]{geometry}

\addbibresource{viagem.bib}

\title{Viagem à volta do meu quarto} \date{10 de
  Dezembro de 2015} \author{João Távora \\Faculdade de Belas Artes da
  Universidade de Lisboa}

\pretitle{\begin{flushright}\LARGE\sffamily}
\posttitle{\par\end{flushright}\vskip 0.5em}
\predate{\begin{flushleft}\large\scshape}
\postdate{\par\end{flushleft}}

\begin{document}

\maketitle

\section{Introdução}

``Viagem à volta do meu quarto'' é o título que Xavier de Maistre deu ao livro que escreveu em 1795 e no qual descreve o período de 42 de prisão domiciliária que lhe é imposto por envolvimento num duelo. O livro é um diário desse cárcere, escrito em jeito de romance de viagems. Trata-se de uma paródia da literatura deste género, particularmente estimada na época em que o ``Grand Tour'' europeu era um ritual iniciático fundamental na vida de qualquer jovem aristocrático. A este livro, de Maistre somou em 1825 outro, chamado ``Expedição noctura à volta do meu quarto''. Ambos se encontram estruturados em tantos capítulos quantos dias duraram as respectivas viagens.

Param além do elemento cómico, estes livro influenciaram artistas e escritores, e são uma defesa da imaginação e da intimidade intelectual como fontes de descoberta. Este texto tem como objecto os aspectos da obra de Maistre que possam encontrar pa

\section{Desenvolvimento}

\subsection{Da paródia à inquietação filosófica}

Nas primeiras linhas de Maistre anuncia imodestamente que ``aparece de repente no mundo erudito'' com um ``livro de descobertas na mão''. Viajar no próprio quarto é ``um meio seguro contra o tédio'' e a cómoda e democrática forma de viajar: serve os indolentes, os indigentes, os medrosos.

O principal dispositivo lúdico é a insólita precisão do simulacro de viagem: de forma perfeitamente convincente descrevem-se levantamentos topológicos, reuniões, separações e imprevistos. Habilmente, expõe-se na descrição de cada um acontecimentos o ridículo do viajante fechado no seu quarto, e, de alguma forma, algum desse ridículo transfere-se para o referente real.

No entanto, ao contrário da paródia habitual, e pela disparidade inusitada da comparação, o segundo transforma-se facilmente em inquietação filosófica. Desta forma, que chega quase a ser sinistra - não faz diferença nenhuma se viajamos pelo nosso quarto ou por qualquer país exótico - o livro transforma-se em mais que um \emph{pastiche} da literatura de viagens.

O livro é tanto mais rico quanto a exploração da ideia filosófica de viagem é apenas uma das suas possiblidades de leitura. Quanto a ela, encontra-se na obra livro tanto a sua negação filosófica quanto a sua sublimacão ao essencial. No prefácio à edição portuguesa (Mexia, 2015) ressalta-se, por exemplo, a dimensão política do livro escrito à sombra da ``tirania'' da Revolução Francesa.

A todos estes estratos soma-se o outro que que advém do facto de, em última instância, o tema do livro ser a construção dele mesmo, na sua bizarra natureza.

Depois do começo triunfalista, é logo ao início que morre a vã esperança de alguma estabilidade estrutura que provesse da promessa que o protagonista faz em descrever em 42 capítulos os 42 dias do seu cárcere. É como se houvesse um eco recíproco entre todas as agitações metafísicas ou sentimentais e a própria forma. Há capítulos puramente indolentes e outros prolíficos. Há capítulos muito curtos onde nada mais se faz do que esperar pelo próximo, capítulos em que se anunciam futuros capítulos. É como cada capítulo do livro é um atentado contra todos os outros.

Este é 

Na ``Expecição nocturna...'' o próprio sente finalmente a necessidade consagrar um capítulo ao próprio processo de escrita:

\begin{quote}
  Tinha uma velha parente, mulher de muita inteligência, cuja conversa era das mais interessantes; a sua memória, porém, inconstante e fértil a um tempo, fazia-a passar muitas vezes de um espisídio a outro e de uma divigação a outra, a ponto de ver-se obrigada a implorar a ajuda de quem a ouvia: ``Que é que vos queria contar?'', dizia ela, e muitas vezes também os seus ouvintes se tinham esquecido, o que deixava toda a gente num apuro difícil de expressar. Ora, pôde notar-se que o mesmo acidente se verifica frequentemente nas minhas narrações, e devo convir, efectivamente, que o plano e a ordem da minha viagem são exactamente decalcados da ordem e do plano das conversas da minha tia; mas não peço o auxílio de ninguém, pois reparei que o assunto volta por si mesmo e no momento que menos espero.
\end{quote}


outro, objecto de estudo deste texto: o enigma da própria escrita de um livro de tão bizarra natureza. 



circumnavegações

objectivo do trabalho. manual de criação plástica.

\section{Luzes, tecidos, torradas}

\section{O acidente}

Susan Sontag

\section{O outeiro}

Martin Heidegger

sapatos de camponês

``Há tanta suavidade em nada se dizer / E tudo se entender — '' Pessoa

``Há muitas maneiras sérias de não dizer nada, mas só a poesia é verdadeira'' Manoel de Barros

``O nada onde está tudo''

\section{Outros}

Giotto
Duccio
Simone Martini
Pietro Lorenzetti e Ambrosio Lorenzetti (o intelectual)

\printbibliography[heading=bibliography,title={Bibliografia}]

\end{document}

%% Local Variables:
%% coding: utf-8
%% End:
