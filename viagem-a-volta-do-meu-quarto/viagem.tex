\documentclass[12pt]{article}
\usepackage[portuguese]{babel}
\usepackage{csquotes}
\usepackage{graphicx}
\usepackage{adjustbox}
\usepackage{lipsum}



%%%% Espaçamento por omissão
%%%%
\usepackage{setspace}
\doublespacing


%%%% Fonts
%%%%
\usepackage{fontspec}
\setmainfont[]{Times New Roman}


%%%% Citacoes
%%%%
\newenvironment{citacao}
{\begin{quote}
    \begin{small}
      \itshape
      \linespread{1}
      %% \setlength{\rightmargin}{\leftmargin}
      %% \setlength{\leftmargin}{2\leftmargin}
}
{\end{small}\end{quote}}



%%%% Margins
%%%% 
\usepackage[margin=2cm,left=3cm]{geometry}


%%%% Bibliography
%%%% 
\usepackage[backend=biber,autolang=other,
  bibencoding=utf8,style=authoryear-ibid,
  ibidtracker=true]{biblatex}
\addbibresource{viagem.bib}


%%%% Sections como Queirós
%%%%
\usepackage{sectsty}

\renewcommand\thesection{}
\renewcommand\thesubsection{\thesection\arabic{subsection}.}

%% OMG, stolen from http://tex.stackexchange.com/questions/80113/hide-section-numbers-but-keep-numbering#80114
\makeatletter
\def\@seccntformat#1{\csname #1ignore\expandafter\endcsname\csname the#1\endcsname\quad}
\let\sectionignore\@gobbletwo
\let\latex@numberline\numberline
\def\numberline#1{\if\relax#1\relax\else\latex@numberline{#1}\fi}
\makeatother

\allsectionsfont{\normalsize\raggedright}
\subsectionfont{\normalsize\leftskip\leftskip\leftskip\leftskip\leftskip\leftskip}

\usepackage{titlesec}
\titlespacing*{\section}{0pt}{0.7\baselineskip}{0\baselineskip}
\titlespacing*{\subsection}{1cm}{0\baselineskip}{0\baselineskip}
\titlespacing*{\subsubsection}{0pt}{0\baselineskip}{0\baselineskip}

\usepackage{indentfirst}
\setlength{\parindent}{1cm}


%% Sumário
%%

\usepackage{tocloft}


\setlength{\cftbeforesubsecskip}{-2ex}
\setlength{\cftbeforesecskip}{-2ex}
\setlength{\cftsecindent}{-15pt}

\renewcommand{\cfttoctitlefont}{\bfseries\small}
\setlength{\cftaftertoctitleskip}{-2ex}

\renewcommand{\cftsecfont}{\roman}

%% \renewcommand{\cftsecpagefont}{\roman\cftsecleader}


%% \newcommand{\cftsecfont}{}
%% \newcommand{\cftsecaftersnum}{}
%% \newcommand{\cftsecaftersnumb}{}
%% \newcommand{\cftsecleader}{\cftdotfill{\cftsecdotsep}}
%% \newcommand{\cftsecdotsep}{\cftdotsep}
%% \newcommand{\cftsecpagefont}{}
%% \newcommand{\cftsecafterpnum}{}



%% Título
%% 
\usepackage{titling}
\setlength{\droptitle}{-2cm}
\pretitle{\begin{center}\normalsize}
\posttitle{\end{center}}
\preauthor{\begin{center}\normalsize}
\postauthor{\end{center}}
\predate{\begin{center}\normalsize}
\postdate{\end{center}}

\title{\large{Viagem à Volta do Meu Quarto}}
\author{\vspace*{-3ex}João Távora, nº 6311}
\date{\vspace*{-3ex}Curso de Desenho, 3º Ano\\\vspace*{-1ex}FBAUL, 2016}
%% 


%% Page numbers
%%
\usepackage{fancyhdr}
\pagestyle{fancy} % Turn on the style
\fancyhf{} % Start with clearing everything in the header and footer
% Set the right side of the footer to be the page number
\fancyfoot[R]{\thepage}
\renewcommand{\headrulewidth}{0pt}

% Redefine plain style, which is used for titlepage and chapter beginnings
% From http://tex.stackexchange.com/a/30230/828
\fancypagestyle{plain}{%
    \renewcommand{\headrulewidth}{0pt}%
    \fancyhf{}%
    \fancyfoot[R]{\thepage}%
}


%% Bibliografia
%%



%% Início do documento
%% 
\begin{document}

%%%% DONE: le tértre, o que é?
%%%% TODO: Bibliografia, maketitle
%%%% TODO: Muitas referências a acidentes
%%%% DONE: Imagem quadro do van gogh
%%%% TODO: Revolução realista (dignidade dos temas esquecidos da realidade)
%%%% DONE: o Quarto como autoretrato enquanto ausência

\maketitle

\renewcommand\contentsname{Sumário}
\tableofcontents

\begin{spacing}{1.0}
\begin{small}
  \section{\fontsize{11}{13}\selectfont{Resumo}}

    \noindent Estuda-se \emph{Viagem à Volta do Meu Quarto}, pequeno
    livro de 1795 de Xavier de Maistre, paródia da literatura de
    viagens popular na altura, procurando nele pistas sobre os
    processos e práticas artística da Pintura e do Desenho, destapando
    a sua influência noutras obras, desde a sua publicação até ao
    presente, propondo paralelos entre o texto e alguns autores da
    filosofia, da poesia e das artes plásticas.\\
    \noindent\textbf{Palavras chave:} de Maistre, pintura, desenho,
    viagem, acidente
    
\end{small}

\end{spacing}
\vspace*{-1ex}

\section{Introdução}

\emph{Viagem à Volta do Meu Quarto} é o título que Xavier de Maistre
deu ao livro que escreveu em 1792 e no qual descreve o período de 42
dias de prisão domiciliária que lhe é imposto por envolvimento num
duelo. O livro é um diário desse cárcere real, escrito em jeito de
romance de viagems, num percurso imaginado e acidentado pelas memórias
que o espaço do quarto evoca no protagonista. Trata-se de uma paródia
da literatura deste género, particularmente estimada na época em que o
\emph{Grand Tour} europeu era um ritual iniciático na vida de qualquer
jovem aristocrático. A este livro, de Maistre mais tarde outro,
publicado em 1825, chamado ''Expedição noctura à volta do meu
quarto''. Ambos se encontram estruturados em tantos capítulos quantos
dias duraram as respectivas viagens.

Para além de comédias, estes livros são uma defesa da imaginação e da
intimidade afectiva como fontes de descoberta intelectual e
criativa. Escritos durante a turbulência da revolução francesa, na
eclosão da modernidade, influenciaram escritores e artistas plásticos
até ao presente. O presente texto tem como objecto destapar e
desenvolver aspectos particulares da obra de de Maistre que possam
constituir incursões nos processos das artes plásticas, sobretudo na
Pintura e no Desenho.

\section{Desenvolvimento}

\subsection{Da paródia à inquietação filosófica}

Nas primeiras linhas, de Maistre anuncia imodestamente que ''aparece
de repente no mundo erudito'' com um ''livro de descobertas na
mão''. Viajar no próprio quarto é ''um meio seguro contra o tédio'' e
a cómoda e democrática forma de viajar: serve os indolentes, os
indigentes, os medrosos.

O simulacro de viagem, executado com insólita precisão, constitiu-se
como principal engenho lúdico. Descreve-se convincentemente um
levantamento topológico do espaço do quarto, viagens a cavalo do
parapeito da janela, reuniões, separações e imprevistos. Cada relato
acaba por expõr também a banalidade desse quarto e o carácte,
frequentemente frívolo e ridículo do seu habitante vestido de
\emph{roupão de viagem}, viajante fechado no quarto.

A paródia consiste em que algum desse ridículo se transfere para o
referente e se transforme em comentário trocista da grandiosidade dos
relatos de viagens reais. Mas, da disparadidade absurda da comparação,
estabelece-se também uma certa inquietação filosófica. Nessa dimensão
quase sinistra - não faz diferença nenhuma se viajamos pelo nosso
quarto ou por qualquer lugar exótico - a viagem de de Maistre
transforma-se em muito mais que um \emph{pastiche} da literatura de
viagens.

O livro é tanto mais rico quanto essa exploração da ideia de viagem é
apenas uma das suas possiblidades de leitura. Quanto a ela,
encontra-se na obra livro tanto a sua negação categórica
\cite{queiros} quanto a afirmação da sua essência. No entanto,
encontram-se outros estratos temáticos: no prefácio à edição
portuguesa, Pedro Mexia ressalta, por exemplo, o a ideia de ''exílio
íntimo'', e a dimensão política do livro escrito à sombra da
''tirania'' da Revolução Francesa (\cite[p.10]{demaistre}).

Refira-se que, se nessas leituras há agitações do pensamento, elas são
de natureza mais latente, e convém distingui-las, mas não demasiado,
das de natureza mais explícita, como é o caso da repetida oposição
entre a alma do protagonista e a sua existência material, \emph{o
  animal}, numa espécie de falseto neoplatónico que o autor detalha no
capítulo VI, aquele que é destinado ''tão-só e apenas aos
metafísicos'' (\cite[p.25]{demaistre}).

\subsection{Das torradas à Pintura}

De Maistre investe bastante da sua viagem em considerações sobre as
artes. Não se pode dizer que a Pintura ocupe nelas um lugar cimeiro:
apesar de a descrever como ''arte sublime'' (\cite[p.27]{demaistre})
trata-se de apenas de uma promoção pouco sincera, um mero
galanteio. Ainda assim, quando não se está a inquietar filosoficamente
ou a queimar o seu \emph{animal} na tenaz da fogueira, podemos
encontrar de Maistre evocando ideias visuais.

Descrevem-se os padrões abstractos de luz e sombra que as árvores lá
fora produzem na parede (\cite[p.23]{demaistre}), a cor perfeita da
cama (\cite[p.53]{demaistre}), contemplam-se retratos e estampas
(\cite[p.34]{demaistre}), dialoga-se com Rafael
(\cite[p.65]{demaistre}), invoca-se Apeles
(\cite[p.69]{demaistre}). Ao examinar a correspondência antiga na sua
escrivaninha, apreciam-se as ''linhas traçadas'', na caligrafia
executada pela ''mão conduzida pelo coração'' (\cite[p.84]{demaistre})
Cada objecto do quarto é manejado e vivido com preocupação pelas sua
forma e textura. Mantém-se com cada um deles, desde as vigas do
telhado, à cafeteira, à pirâmide de torradas, uma relação de
reverência. É é quase sempre a pretexto deles que se iniciam ou
terminam etapas da viagem.

A propósito dessa reverência, não escaparam a Guiliana Bruno,
professora de estudos visuais em Harvard, ligações entre a
\emph{Viagem} e as artes plásticas. Bruno descreve o museu Marés, que
o escultor Frederic Marés (1893-1991) fundou e baptizou de \emph{Museu
  Sentimental}. A autora fala do ''espectáculo das coisas que não
transportam outro valor senão o seu poder emocional - objectos
transformados em narrativas'' e de como esse museu, que contém apenas
objectos de uso pessoal de vários tipos, infuenciou por diversas vezes
a arte contemporânea. Nessa medida, o quarto de de Maistre situa-se no
coração deste museu; o seu livro é uma ''viagem sentimental'' na
''eclosão da modernidade'' (\cite[p.133]{bruno2002atlas}).

Noutra ocasião, Bruno apresenta o trabalho \emph{Atlas} de Gehrard
Richter como um ''arquivo de espaço íntimo'', um ''movimento
narrativo'' nascido da ''geografia pessoal'' do autor
(\cite[p.334]{bruno2002atlas}). Para a autora, esse movimento é a
''nova forma de viajar'' inaugurada por de Maistre.

Durante o século XX, de Egon Schiele a pintores menos conhecidos como
John Bratby, podemos contemplar os quartos que os artistas habitaram
em pinturas desses espaços. No entanto, talvez aquelas que Vincent Van
Gogh fez do seu modesto quarto em Arles (Figura \ref{fig:1}) sejam os
mais célebres\footnote{A imagem de um destes quadros faz precisamente
  a capa de uma edição anterior da ''Viagem'' (\cite{hedra})} e talvez
os primeiros do género.

\begin{figure}
  \centering\includegraphics[height=0.4\textheight,keepaspectratio] {slaapkamer.jpg}
  \caption{Quarto de Van Gogh em Arles}
  \label{fig:1}
\end{figure}

Se não sabemos se Van Gogh conheceu de Maistre, talvez possamos dizer
que, fechado nesse quarto por doença (\cite{goghroom}), concebeu estes
quadros em condições muito semelhantes às do escritor - e também
razoalvemente inquieto - pelo menos a julgar por uma carta que remeteu
ao seu irmão Theo, onde tenta descrever a composição: ''Enfim la vue
du tableau doit reposer la tête''. Ao reler a carta, o pintor
acrescenta de um modo enigmático e em letra mais pequena ''ou plutôt
la imagination'' (\cite{goghcarta}).

Antes de Van Gogh, podem encontrar-se outras representações do espaço
do artista (no seu atelier, por exemplo), mas o tema nunca é o próprio
quarto ou a carga de intimidade que ele transporta. Van Gogh recria o
quarto como uma espécie de \emph{autoretrato enquanto ausência}, algo
que de Maistre também fez, na forma escrita, quase cem anos antes.

\subsection{Do outeiro ao acidente}

O capítulo XII da ''Viagem...'' é formalmente o mais insólito. Ei-lo,
integralmente (\cite[p.36]{demaistre}):

\begin{center}
\begin{verbatim}
                     . . . . . . . . . . . . . . .
                     . . . . . . . . . . . . . . .
                     . . . . . o outeiro . . . . .
                     . . . . . . . . . . . . . . .
                     . . . . . . . . . . . . . . .
                     . . . . . . . . . . . . . . .
\end{verbatim}
\end{center}

''O outeiro'' de que fala de Maistre é o local que recordou, no
capítulo anterior em que limpava o pó a um retrato, de ver pela última
vez um amor predilecto.

A emoção é tão forte que a ''imagem daquele outeiro'' acaba por
eclipsar todos os outros ''pensamentos desconexos'', bem como um
capítulo inteiro (''os meus esforços são baldados [...] é uma etapa
militar'' \cite[p.36,p.39]{demaistre})

De Maistre concebe \emph{o outeiro}, uma pequena elevação de terreno,
possivelmente o mais modesto dos acidentes geográficos, numa fonte de
ausência e de assombro. Não só convida o leitor a imaginá-lo como
quase o obriga, mas do quanto ao local em si não sabemos quase
nada. Enquanto outros seus contemporâneos românticos escreverão sobre
montanhas nevadas ou planícies verdejantes, de Maistre abriga tudo no
\emph{outeiro}, ou seja, \emph{em nada}. E este paradoxo, perfeitamente
consistente com o da viagem imóvel, é talvez ainda mais poderoso.

Martin Heidegger, que por coincidência também escrevia em retiro
(\cite{heideggerhut}) concebe uma ideia semelhante n'\emph{A Origem da
  Obra de Arte}, e precisamente acerca de uns sapatos de camponês
pintados pelo pintor de quartos que apontámos no capítulo anterior
(\cite[p.24]{heidegger}):

\singlespacing

\begin{citacao}
  A partir da pintura de Van Gogh não podemos sequer estabelecer onde
  se encontram estes sapatos.[...] Um par se sapatos de camponês e
  nada mais. E todavia...  Na escura abertura do interior gasto dos
  sapatos, fita-nos a dificuldade e o cansaço dos passos do
  trabalhador.
\end{citacao}

\doublespacing

O \emph{nada} é uma curiosa predilecção de algum poetas de língua
portuguesa do século XX, como para Fernando Pessoa (''Há tanta
suavidade em nada se dizer / E tudo se entender'' — (\cite{pessoa}))
ou para o poeta brisileiro Manoel de Barros.

Este último tem um poema, no seu \emph{Livro das Ignorâças}, a
propósito de uma enseada, que se pode ouvir \emph{on-line} dito pelo
próprio autor em (\cite{vidabreve},\cite{manoel}).

\singlespacing

\begin{citacao}
  O rio que fazia uma volta atrás de nossa casa\\
era a imagem de um vidro mole que fazia uma\\
volta atrás de casa.\\
Passou um homem depois e disse: Essa volta\\
que o rio faz por trás de sua casa se chama\\
enseada.\\
Não era mais a imagem de uma cobra de vidro\\
que fazia uma volta atrás de casa.\\
Era uma enseada.\\

Acho que o nome empobreceu a imagem. \\
\end{citacao}

\doublespacing

Inversamente a de Maistre, aqui é o nome que devora a imagem, mas isso
não coloca nenhum entrave a que estes versos se encontrem muito
próximos do lugar poético do \emph{outeiro}. Ambros afirmam a elipse,
a supressão e a ausência, não só como figuras de estilo mas também
como fontes de imagens.

Nos dois casos, a ideia de \emph{acidente geográfico} parece não ser
casual; um \emph{acidente} não é um acontecimento qualquer. No limite
não pode ser descrito - da mesma forma que, ainda que de Maistre o
tenha tentado, quando examinava velha correspondência, não conseguiu
descrever um traço ou uma pincelada sem se
emocionar. (\cite[p.84]{demaistre})

A palavra \emph{acidente} abunda na pequena viagems de de Maistre
(\cite[p.71, p.115, p.139, p.182, p.191, p.195,
  p. 205]{demaistre}). Também abundam as tempestades e derivas, como
as que fazem que, inexplicavelmente, os ''capítulos terminem sempre
num tom sinistro'' (\cite[p.61]{demaistre}). Por fim, é o próprio
autor que, na \emph{Expedição nocturna...}, sente a necessidade
consagrar um capítulo à sua estratégia compositiva acidental
(\cite[p.182]{demaistre}):

\singlespacing

\begin{citacao}
  Tinha uma velha parente, mulher de muita inteligência, cuja conversa
  era das mais interessantes; a sua memória, porém, inconstante e
  fértil a um tempo, fazia-a passar muitas vezes de um episódio a
  outro e de uma divigação a outra, a ponto de ver-se obrigada a
  implorar a ajuda de quem a ouvia: ''Que é que vos queria contar?'',
  dizia ela, e muitas vezes também os seus ouvintes se tinham
  esquecido, o que deixava toda a gente num apuro difícil de
  expressar. Ora, pôde notar-se que o mesmo acidente se verifica
  frequentemente nas minhas narrações, e devo convir, efectivamente,
  que o plano e a ordem da minha viagem são exactamente decalcados da
  ordem e do plano das conversas da minha tia; mas não peço o auxílio
  de ninguém, pois reparei que o assunto volta por si mesmo e no
  momento que menos espero.
\end{citacao}

\doublespacing

\section{Conclusão}

Em última instância, o tema principal da \emph{Viagem à Volta do Meu
  Quarto} é a construção do próprio livro, ou o apuro contínuo em que
isso decorre. Desde cedo se defraudam as espectativas de um mínimo de
estabilidade estrutural feita da coincidência entre os 42 dias de
cárcere e os 42 capítulos, como se de uma viagem minimamente normal se
pudesse vir a tratar, em que se percorreria o espaço físico do quarto
com lentidão ou minúcia absurdas.

Pelo contrário, encontram-se as agitações metafísicas ou sentimentais
do protagonista na própria forma do livro. A única consistência que
sobrevive da equação capítulo-dia é a disparidade entre eles: como
dias, há capítulos puramente indolentes e outros enérgicos e
resolutos. Há capítulos muito curtos onde nada mais se faz do que
esperar pelo próximo ou evocar a memória de capítulos passados ou
perdidos.

Cada capítulo do livro é um relato da sua própria vida efémera,
abandonado no último segundo antes de ruir em escombros, como um
Desenho. E cada capítulo é um atentado contra todos os outros.

Assim, aos estratos de viagem imóvel, de paródia cultural, de sátira
política e de exílio sentimental pode-se somar outro, de natureza
puramente estética: o livro é um guia para a espeleologia da
imaginação, um manual de composição cujos engenhos principais são o
paradoxo, o acidente, o desastre iminente. Desta forma, retiram-se
consequências para a criação plástica em qualquer linguagem, mas
talvez mais para o Desenho e a Pintura.

Mas se, nas viagens que fazemos através dele, o quarto de Maistre nos
fita como um tratado de Pintura ou um caderno de Desenhos, outras
viagens serão possíveis. Aliás, como se observa no prefácio, as
diferenças entre a \emph{Viagem à Volta do Meu Quarto} e a
\emph{Expedição nocturna}, mostram precisamente que uma viagem, como
um acidente, é uma coisa irrepetível.

\renewcommand*{\bibfont}{\footnotesize}

\printbibliography[heading=bibliography,title={\hspace{1cm}Referências}]

\end{document}

%%%% Local Variables:
%%%% coding: utf-8
%%%% tex-open-quote: "''"
%%%% End:
