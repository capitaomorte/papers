\documentclass[12pt]{article}
\usepackage{fontspec}
\usepackage{titling}
\usepackage[portuguese]{babel}
\renewcommand{\baselinestretch}{1.5} 
\usepackage{csquotes}
\usepackage{graphicx}
\usepackage{adjustbox}
\usepackage{lipsum}
\usepackage[backend=biber,autolang=other,
  bibencoding=utf8,style=authoryear-ibid,
  ibidtracker=true]{biblatex}

\setmainfont[]{Times New Roman}

\usepackage[margin=2cm,left=3cm]{geometry}

\addbibresource{viagem.bib}

\title{Viagem à volta do meu quarto}
\date{10 de Dezembro de 2015}
\author{João Távora \\Faculdade de Belas Artes da Universidade de Lisboa}

\begin{document}
%% TODO: le tértre, o que é?
%% TODO: Bibliografia, maketitle
%% TODO: Muitas referências a acidentes
%% TODO: Imagem quadro do van gogh

\emph{Viagem à volta do meu quarto}\\
\emph{João Távora}\\
\emph{Faculdade de Belas Artes da Universidade de Lisboa}\\
\emph{10 de Dezembro de 2015}\\

\section{Resumo}

Estuda-se ``Viagem à volta do meu quarto'', pequeno livro de 1795 de Xavier de Maistre, paródia da literatura de viagens popular na altura, procurando nele pistas sobre os processos criativos ligados principalmente à prática artística da Pintura e do Desenho, destapando a sua influência noutras obras, desde a sua publicação até ao presente, propondo paralelos entre o texto e alguns autores da filosofia, da poesia e das artes plásticas.

\section{Introdução}

``Viagem à volta do meu quarto'' é o título que Xavier de Maistre deu ao livro que escreveu em 1795 e no qual descreve o período de 42 dias de prisão domiciliária que lhe é imposto por envolvimento num duelo. O livro é um diário desse cárcere real, escrito em jeito de romance de viagems, num percurso imaginado e acidentado pelas memórias que o espaço desse quarto evoca no protagonista. Trata-se de uma paródia da literatura deste género, particularmente estimada na época em que o ``Grand Tour'' europeu era um ritual iniciático na vida de qualquer jovem aristocrático. A este livro, de Maistre somou outro em 1825, chamado ``Expedição noctura à volta do meu quarto''. Ambos se encontram estruturados em tantos capítulos quantos dias duraram as respectivas viagens.

Param além de comédias, estes livros são uma defesa da imaginação e da intimidade afectiva como fontes de descoberta intelectual e criativa. Escritos durante a turbulência da revolução francesa, na eclosão da modernidade, influenciaram escritores e artistas plásticos até ao presente. O presente texto tem como objecto destapar alguns aspectos particulares da obra de Maistre que possam constituir incursões nos processos das artes plásticas, sobretudo na Pintura e no Desenho.

\section{Desenvolvimento}

\subsection{Da paródia à inquietação filosófica}

Nas primeiras linhas de Maistre anuncia imodestamente que ``aparece de repente no mundo erudito'' com um ``livro de descobertas na mão''. Viajar no próprio quarto é ``um meio seguro contra o tédio'' e a cómoda e democrática forma de viajar: serve os indolentes, os indigentes, os medrosos.

O principal dispositivo lúdico é o simulacro de viagem, excutado com insólita precisão. Descreve-se convincentemente um levantamento  topológico do espaço do quarto, viagens a cavalo do parapeito da janela, reuniões, separações e imprevistos. Mas cada relato acaba por expõr também a banalidade desse quarto e do carácter indolente e ridículo do seu habitante vestido de \emph{roupão de viagem}, viajante fechado no quarto.

A paródia consiste em que algum desse ridículo se transfere para o referente e e troce também da grandiosidade dos relatos de viagens reais. A disparadidade absurda da comparação faz ainda com que esse comentário se transforme em inquietação filosófica. Nessa dimensão quase sinistra - não faz diferença nenhuma se viajamos pelo nosso quarto ou por qualquer lugar exótico - a viagem de de Maistre transforma-se em muito mais que um \emph{pastiche} da literatura de viagens.

O livro é tanto mais rico quanto essa exploração da ideia de viagem é apenas uma das suas possiblidades de leitura. Quanto a ela, encontra-se na obra livro tanto a sua negação categórica (Queirós, 2015) quanto a afirmação da sua essência. Mas encontram-se outros estratos temáticos: no prefácio à edição portuguesa, Pedro Mexia ressalta, por exemplo, o a ideia de ``exílio íntimo'', e a dimensão política do livro escrito à sombra da ``tirania'' da Revolução Francesa.

Refira-se que se nessas leituras há agitações do pensamento, elas são de natureza mais latente, convém distingui-las, mas não demasiado, das de natureza mais explícita, como é o caso da repetida oposição entre a alma do protagonista e a sua existência material, \emph{o animal}, uma espécie de falseto neoplatónico que o autor detalha no capítulo VI, aquele que é destinado ``tão-só e apenas aos metafísicos''.

\subsection{Das torradas à Pintura}

De Maistre inviste bastante da sua viagem em considerações sobre as artes, mas não se pode dizer que a Pintura ocupe nelas um lugar cimeiro, pelo menos não de forma manifesta. Ainda assim, quando não se está a inquietar filosoficamente ou a queimar o seu animal na tenaz da fogueira, podemos encontrar de Maistre evocando ideias visuais.

Descrevem-se os padrões abstractos de luz e sombra que as árvores lá fora produzem na parede, a cor perfeita da cama, contemplam-se retratos e estampas, dialoga-se com Rafael (ONDE?), invoca-se Apeles (ONDE?). Ao examinar a correspondência antiga na sua escrivaninha, apreciam-se as ``linhas traçadas'', na caligrafia executada pela ``mão conduzida pelo coração''. Manejam-se os objectos com preocupação pelas suas formas e texturas. Acima de tudo, mantém-se com cada objecto mundano, desde as vigas do telhado, à cafeteira, à pirâmide de torradas, uma relação íntima. Há por estes objectos uma certa reverência e e quase sempre a pretexto deles que se iniciam etapas da viagem.

A propósito dessa reverência, não escaparam a Guiliana Bruno, professora de estudos visuais em Harvard, ligações entre a ``Viagem...'' e as artes plásticas. Bruno descreve o museu Marés, que o escultor Frederic Marés (1893-1991) fundou e baptizou de ``Museu Sentimental''. A autora fala do ``espectáculo das coisas que não transportam outro valor senão o seu poder emocional - objectos transformados em narrativas'' e de como esse museu, que contém apenas objectos de uso pessoal de vários tipos, infuenciou por diversas vezes obras de arte contemporânea. Nesta medida o quarto de de Maistre situa-se no coração deste museu, o seu livro é uma ``viagem sentimental'' na ``eclosão da modernidade'' (\cite[p.133]{bruno2002atlas}). Noutra ocasião, Bruno descreve o trabalho ``Atlas'' de Gehrard Richter como um ``arquivo de espaço íntimo'' que se radica em grande parte na obra de Maistre. (\cite[p.254]{bruno2002atlas}).

Durante o século XX, de Egon Schiele a pintores menos conhecidos como John Bratby, podemos contemplar os quartos que os artistas habitaram em pinturas desses espaços. No entanto, talvez aquelas que Vincent Van Gogh fez do seu modesto quarto em Arles \ref{fig:1} sejam os mais célebres, e talvez os primeiros do género. \footnote{A imagem de um destes quadros faz precisamente a capa de uma edição anterior da ``Viagem'' (QUAL?)}.

\begin{figure}
  \label{fig:1}
  \caption{Quarto de Van Gogh em Arles}
\end{figure}

Não sabemos se Van Gogh conheceu de Maistre, mas talvez possamos dizer que, fechado nesse quarto por doença (\cite{goghroom}), terá concebido estes quadros em condições muito semelhantes às do escritor - e também razoalvemente inquieto - pelo menos a julgar por uma carta que remeteu ao seu irmão Theo, onde tenta descrever a composição: ``Enfim la vue du tableu doit reposer la tête''. Ao reler a carta acrescenta de um modo igualmente enigmático e em letra mais pequena ``ou plutôt la imagination'' (Van Gogh 1988). 

\subsection{Do outeiro ao acidente}

Em última instância, o tema da ``Viagem à volta do meu quarto'' é a construção do próprio livro. Apesar das promessas do escritor, defrauda-se logo o leitor incauto que esperasse o mínimo de estabilidade estrutural feita da coincidência entre os 42 dias de cárcere e os 42 capítulos, como se de uma viagem minimamente normal se pudesse vir a tratar, em que se percorresse o espaço físico do quarto com lentidão ou minúcia absurdas.

Pelo contrário, a única que subsiste da analogia capítulo-dia é a disparidade entre todos eles. As agitações metafísicas ou sentimentais do protagonista ecoam-se na forma do livro: como dias, há capítulos puramente indolentes e outros enérgicos e resolutos. Há capítulos muito curtos onde nada mais se faz do que esperar pelo próximo ou se evoca o capítulos passados. 

Cada capítulo do livro é um esboço inacabado que da sua própria vida curta, abandonada displicentemente em desastre, como um Desenho, dos quais restam uns escombros de ideia. E cada capítulo é um atentado contra todos os outros.

O capítulo XII da ``Viagem...'' é sem dúvida o mais insólito. Ei-lo, integralmente:

\begin{verbatim}
  . . . . . . . . . . . . . . .
  . . . . . . . . . . . . . . .
  . . . . . o outeiro . . . . .
  . . . . . . . . . . . . . . .
  . . . . . . . . . . . . . . .
  . . . . . . . . . . . . . . .
\end{verbatim}

``O outeiro'' de que fala de Maistre é o local que recordou, no
capítulo anterior em que limpava o pó a um retrato, de ver pela última vez um amor predilecto. 

A emoção é tão forte, que esse outeiro acaba por eclipsar todos os outros ``pensamentos desconexos'' um capítulo inteiro (``os meus esforços são baldados [...] é uma etapa militar''). (de Maistre, p???)

De Maistre transforma o outeiro, pequena elevação de terreno, possivelmente o mais modesto dos acidentes geográficos, numa fonte de ausência e de assombro. Não só convida o leitor a imaginá-lo como quase o obriga, enquanto dele se diz nada. Enquanto outros seus contemporâneos românticos escreverão sobre montanhas nevadas ou planícies verdejantes, de Maistre abriga tudo num outeiro, ou seja em nada. E este paradoxo, perfeitamente consistente com o da viagem imóvel, é talvez ainda mais poderoso que ele.

O poder expressivo do ``nada'' é curiosa predilecção de algum poetas de língua portuguesa do século XX, como Fernando Pessoa ou Manoel de Barros \cite{manoel}:

\begin{quote}
  ``Há tanta suavidade em nada se dizer / E tudo se entender — ''
\end{quote}
Fernando pessoa

\begin{quote}
``Há muitas maneiras sérias de não dizer nada, mas só a poesia é verdadeira''
\end{quote}

O poeta brasileiro tem também um poema a propósito de outro acidente geográfico, que se pode ouvir \emph{on-line} dito pelo próprio autor.

\begin{quote}
  O rio que fazia uma volta atrás de nossa casa
era a imagem de um vidro mole que fazia uma
volta atrás de casa.
Passou um homem depois e disse: Essa volta
que o rio faz por trás de sua casa se chama
enseada.
Não era mais a imagem de uma cobra de vidro
que fazia uma volta atrás de casa.
Era uma enseada.

Acho que o nome empobreceu a imagem. 
\end{quote}

  Ainda que aqui seja, inversamente a de Maistre, o nome que devora a imagem, este poema encontra-se muito próximo do lugar poético que o ``outeiro'': afirma a elipse como figura de estilo e a ausência como fonte imagética.

  A própria ideia de \emph{acidente geográfico} não é inocente. Um \emph{acidente} não é um acontecimento qualquer. No limite não pode ser descrito - da mesma forma que, ainda que de Maistre o tenha tentado quando examinava velha correspondência, não se pode descrever um traço ou uma pincelada. 

Martin Heidegger, que por coincidência também escrevia em retiro, concebe uma ideia semelhante n'``A Origem da Obra de Arte'', e precisamente acerca de uns sapatos de camponês pintados pelo pintor de quartos que apontámos no capítulo anterior (\cite[p.xxx?]{heidegger}): 

\begin{quote}
  A partir da pintura de Van Gogh não podemos sequer estabelecer onde se encontram estes sapatos.[...] Um par se sapatos de camponês e nada mais. E todavia...
  Na escura abertura do interior gasto dos sapatos, fita-nos a dificuldade e o cansaço dos passos do trabalhador.
\end{quote} 

Ou seja o nada do outeiro, como abertura escura, não só existe como nos fita e nos interpela.

Por fim, é o próprio de Maistre que, na ``Expedição nocturna...'' sente a necessidade consagrar um capítulo à sua estratégia compositiva acidental (\cite[p.xxx?]{demaistre}):

\begin{quote}
  Tinha uma velha parente, mulher de muita inteligência, cuja conversa era das mais interessantes; a sua memória, porém, inconstante e fértil a um tempo, fazia-a passar muitas vezes de um episódio a outro e de uma divigação a outra, a ponto de ver-se obrigada a implorar a ajuda de quem a ouvia: ``Que é que vos queria contar?'', dizia ela, e muitas vezes também os seus ouvintes se tinham esquecido, o que deixava toda a gente num apuro difícil de expressar. Ora, pôde notar-se que o mesmo acidente se verifica frequentemente nas minhas narrações, e devo convir, efectivamente, que o plano e a ordem da minha viagem são exactamente decalcados da ordem e do plano das conversas da minha tia; mas não peço o auxílio de ninguém, pois reparei que o assunto volta por si mesmo e no momento que menos espero.
\end{quote}

\section{Conclusão}

Com este texto espero ter demonstrado que à viagem imóvel, à paródia lúdica, à política e ao exílio sentimental, que a todos estes estratos que se econtram na ``Viagem à volta do meu quarto'' se pode somar outro de natureza puramente estética: o livro é uma espécie de manual de composição, um manual de espeleologia da própria imaginação, do qual se retiram consequências para a criação plástica em qualquer das suas linguagem, mas talvez mais para o desenho e a pintura, tanto mais que o a indepedência do tema é afirmada repetidamente pelo paradoxos e acientes.

Mas, se nas viagens que fizemos por ele o quarto de Maistre nos fitou com este aspecto de manual de pintura ou caderno de desenhos, outras viagens serão possíveis. Aliás, como se observa no prefácio, as diferenças entre a ``Viagem...'' e a ``Expedição nocturna...'', mostra precisamente que uma viagem, como um acidente, é uma coisa irrepetível. 

\printbibliography[heading=bibliography,title={Bibliografia}]

\end{document}

%% Local Variables:
%% coding: utf-8
%% End:
