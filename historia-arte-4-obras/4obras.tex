\documentclass{article}
\usepackage[utf8]{inputenc}
\usepackage[portuguese]{babel}
\usepackage{csquotes}
\usepackage{graphicx}
\usepackage{adjustbox}
\usepackage{lipsum}
\usepackage[backend=biber,autolang=other,
  bibencoding=utf8,style=authoryear-ibid,
  ibidtracker=true]{biblatex}

\usepackage[margin=2cm,left=3cm]{geometry}
\fontfamily{<familyname>}\selectfont

\addbibresource{4obras.bib}

\title{Análise Sumária de 4 obras da História da Arte} \date{10 de
  Dezembro de 2015} \author{João Távora \\Faculdade de Belas Artes da
  Universidade de Lisboa}

\begin{document}

\maketitle

\section{Cabeça do Rei Senuseret III}

https://en.wikipedia.org/wiki/Senusret_III

Esta cabeça, que faria parte de uma estatueta de corpo inteiro, é
feita de obsidiana, vidro vulcânico de grande dureza e fragilidade, e
por isso muito dificil de trabalhar. Os materiais duros eram
preferidos pelos Egípcios e eram indicados para as representações
físicas faraónicas, que se pretendiam que conservassem a alma no caso
do corpo mortal se deteriorar. O Faraó tem na sua cabeça, um toucado
glissado encimado pelo ``eucos'', símbolo do poder real.

Império médio
XII dinastia

\section{Torso da Deusa Vénus Anadiómena}

Esta peça em terracota azul envernizada reflecte a influência clássica
própria da época greco-romana entre 30 AC e 300DC.

A deusa, saindo das águas, tem o corpo levemente inclinado para a sua
esquerda, naquilo que se adivinha ser a pose clássica do contraposto,
num movimento ondulante e expressivo que é estranho aos cânones
egípcios. Junto do pescoço, ornamentada com um colar de três voltas,
existe um corte que mostra ser oco o interior da peça, pelo que poderá
tratar-se não apenas de uma estatueta mas de um recipiente
antropomorfo.

\section{Outros}

Giotto
Duccio
Simone Martini
Pietro Lorenzetti e Ambrosio Lorenzetti (o intelectual)

\printbibliography[heading=bibliography,title={Bibliografia}]

\end{document}
